\documentclass[a4paper]{article}
\usepackage{adjustbox}
\usepackage{vntex}
\usepackage{xltabular}
\usepackage[export]{adjustbox}
\usepackage[english,vietnam]{babel}
\usepackage[utf8]{inputenc}
\usepackage[utf8]{vietnam}
\usepackage[utf8]{inputenc}
\usepackage{a4wide,amssymb,epsfig,latexsym,array,hhline,fancyhdr}
\usepackage[normalem]{ulem}
\usepackage{soul}
\usepackage[makeroom]{cancel}
\usepackage{amsmath}
\usepackage{amsthm}
\usepackage{multicol,longtable,amscd}
\usepackage{diagbox}
\usepackage{booktabs}
\usepackage{alltt}
\usepackage[framemethod=tikz]{mdframed}
\usepackage{float}
\usepackage{caption,subcaption}
\usepackage{lastpage}
\usepackage[lined,boxed,commentsnumbered]{algorithm2e}
\usepackage{enumerate}
\usepackage{color}
\usepackage{wrapfig}
\usepackage{graphicx}
\usepackage{lipsum}
\usepackage{array}
\usepackage{tabularx, caption}
\usepackage{multirow}
\usepackage{multicol}
\usepackage{rotating}
\usepackage{graphics}
\usepackage{geometry}
\usepackage{setspace}
\usepackage{epsfig}
\usepackage{tikz}
\usepackage{listings}
\usetikzlibrary{arrows,snakes,backgrounds}
\usepackage[unicode]{hyperref}
\usepackage{tabto}
\hypersetup{urlcolor=blue,linkcolor=black,citecolor=black,colorlinks=true} 
\usepackage{pstcol}
\usepackage[normalem]{ulem}
\usepackage{xcolor}
\usepackage{colortbl}
\usepackage{ragged2e}
\usepackage{indentfirst}
\setlength\parindent{24pt}

\newtheorem{theorem}{{\bf Định lý}}
\newtheorem{property}{{\bf Tính chất}}
\newtheorem{proposition}{{\bf Mệnh đề}}
\newtheorem{corollary}[proposition]{{\bf Hệ quả}}
\newtheorem{lemma}[proposition]{{\bf Bổ đề}}
\newtheorem{exer}{\bf Bài toán}
\theoremstyle{definition}

\def\thesislayout {	% A4: 210 × 297
    \geometry {
        a4paper,
        total={160mm,240mm},  % fix over page
        left=25mm,
        top=30mm,
    }
}
\thesislayout

\definecolor{background}{rgb}{0.95,0.95,0.92}

\lstset{language=R,
    basicstyle=\small\ttfamily,
    backgroundcolor = \color{background},
    stringstyle=\color{teal},
    otherkeywords={0,1,2,3,4,5,6,7,8,9},
    morekeywords={TRUE,FALSE},
    deletekeywords={data,frame,length,as,character},
    keywordstyle=\color{blue},
    commentstyle=\color{teal},
    breaklines = true,
    showstringspaces = false
}

\usepackage{fancyhdr}
\setlength{\headheight}{40pt}
\pagestyle{fancy}
\fancyhead{} %clear all header fields
\fancyhead[L]{
 \begin{tabular}{rl}
    \begin{picture}(25,15)(0,0)
    \put(0,-8){\includegraphics[width=10mm, height=10mm]{Images/logo_hcmut.png}}
    %\put(0,-8){\epsfig{width=10mm,figure=hcmut.eps}}
   \end{picture}&
	%\includegraphics[width=8mm, height=8mm]{hcmut.png} & %
	\begin{tabular}{l}
		\textbf{\bf \ttfamily Trường Đại học Bách Khoa - Đại học Quốc gia TP.Hồ Chí Minh}\\
		\textbf{\bf \ttfamily Khoa Khoa học và Kỹ thuật Máy tính}
	\end{tabular} 	
 \end{tabular}
}
\fancyhead[R]{
	\begin{tabular}{l}
		\tiny \bf \\
		\tiny \bf 
	\end{tabular}  }
\fancyfoot{} % clear all footer fields
\fancyfoot[L]{\scriptsize \ttfamily Bài tập lớn môn Công nghệ phần mềm (CO3001) - Năm học 2023 - 2024}
\fancyfoot[R]{\scriptsize \ttfamily Trang {\thepage}/\pageref{LastPage}}
\renewcommand{\headrulewidth}{0.3pt}
\renewcommand{\footrulewidth}{0.3pt}


%%%
\setcounter{secnumdepth}{4}
\setcounter{tocdepth}{3}
\makeatletter
\newcounter {subsubsubsection}[subsubsection]
\renewcommand\thesubsubsubsection{\thesubsubsection .\@alph\c@subsubsubsection}
\newcommand\subsubsubsection{\@startsection{subsubsubsection}{4}{\z@}%
                                     {-3.25ex\@plus -1ex \@minus -.2ex}%
                                     {1.5ex \@plus .2ex}%
                                     {\normalfont\normalsize\bfseries}}
\newcommand*\l@subsubsubsection{\@dottedtocline{3}{10.0em}{4.1em}}
\newcommand*{\subsubsubsectionmark}[1]{}
\makeatother

\everymath{\color{black}}%make in-line maths symbols blue to read/check easily

\sloppy
\captionsetup[figure]{labelfont={small,bf},textfont={small,it},belowskip=-1pt,aboveskip=-9pt}
%space remove between caption, figure, and text
\captionsetup[table]{labelfont={small,bf},textfont={small,it},belowskip=-1pt,aboveskip=7pt}
%space remove between caption, table, and text

%\floatplacement{figure}{H}%forced here float placement automatically for figures
%\floatplacement{table}{H}%forced here float placement automatically for table
%the following settings (11 lines) are to remove white space before or after the figures and tables
%\setcounter{topnumber}{2}
%\setcounter{bottomnumber}{2}
%\setcounter{totalnumber}{4}
%\renewcommand{\topfraction}{0.85}
%\renewcommand{\bottomfraction}{0.85}
%\renewcommand{\textfraction}{0.15}
%\renewcommand{\floatpagefraction}{0.8}
%\renewcommand{\textfraction}{0.1}
\setlength{\floatsep}{5pt plus 2pt minus 2pt}
\setlength{\textfloatsep}{5pt plus 2pt minus 2pt}
\setlength{\intextsep}{10pt plus 2pt minus 2pt}

\thesislayout

\begin{document}

\begin{titlepage}
\begin{center}
ĐẠI HỌC QUỐC GIA THÀNH PHỐ HỒ CHÍ MINH \\
TRƯỜNG ĐẠI HỌC BÁCH KHOA \\
KHOA KHOA HỌC VÀ KỸ THUẬT MÁY TÍNH \\
\end{center}

\vspace{1cm}

\begin{figure}[h!]
\begin{center}
\includegraphics[width=4cm]{Images/logo_hcmut.png}
\end{center}
\end{figure}

\begin{center}
    \begin{tabular}{c}
        \multicolumn{1}{c}{\textbf{{\huge \color{black} CÔNG NGHỆ PHẦN MỀM (CO3001)}}}\\
        ~~ \\
        \textcolor{black}{\textbf{\hrulefill\hrulefill\hrulefill\hrulefill\hrulefill\hrulefill\hrulefill\hrulefill\hrulefill\hrulefill\hrulefill}}
        \\
                   
                    \textcolor{black}{\textbf{{\huge BÁO CÁO BÀI TẬP LỚN}}}\\\\
                    \textcolor{black}{\textbf{{\large LỚP L01 - NHÓM 12}}}
        ~~\\
        \textcolor{black}{\textbf{\hrulefill\hrulefill\hrulefill\hrulefill\hrulefill\hrulefill\hrulefill\hrulefill\hrulefill\hrulefill\hrulefill}}
        \\
    \end{tabular}
\end{center}

\begin{flushleft}
    \fontsize{14pt}{17pt}\selectfont  
    \textbf{\textsl{\textcolor{black}{\uline{ĐỀ TÀI:}}}}
\end{flushleft}
\begin{center}
    \fontsize{18pt}{17pt}\selectfont 
    \textbf{\textrm{A SMART PRINTING SERVICE}} 
    
    \vspace{12pt}
    
    \textbf{\textrm{FOR STUDENTS AT HCMUT}}

\end{center}

\begin{center}
        \vspace{15pt}
\textbf{GV hướng dẫn: Lê Đình Thuận}
\end{center}

\vspace{10pt}
\textbf{Nhóm sinh viên thực hiện:}

\definecolor{myblack}{RGB}{0,0,0}
\newcolumntype{C}[1]{>{\centering\arraybackslash}m{#1}}
\renewcommand{\arraystretch}{2}
\begin{table}[H]
\centering
\arrayrulecolor{black}
\begin{tabular}{| C{0.75cm} | C{5cm} | C{2cm} |} 
\hline
\textcolor{myblack}{\textbf{STT}} & \textcolor{myblack}{\textbf{Họ và tên}} & \textcolor{myblack}{\textbf{MSSV}} \\ 
\hline
\textcolor{myblack}{1} & \textcolor{myblack}{Đinh Vũ Hà} & \textcolor{myblack}{2113269} \\ \hline
\textcolor{myblack}{2} & \textcolor{myblack}{Nguyễn Hoài Khang} & \textcolor{myblack}{2111453} \\ \hline
\textcolor{myblack}{3} & \textcolor{myblack}{Nguyễn Quang Minh} & \textcolor{myblack}{2111753} \\ \hline
\textcolor{myblack}{4} & \textcolor{myblack}{Từ Mai Thế Nhân} & \textcolor{myblack}{2114277} \\ \hline
\textcolor{myblack}{5} & \textcolor{myblack}{Cù Hoàng Nguyễn Sơn} & \textcolor{myblack}{2112185} \\ \hline
\textcolor{myblack}{6} & \textcolor{myblack}{Đinh Đào Quốc Thịnh} & \textcolor{myblack}{2012107} \\ \hline
\textcolor{myblack}{7} & \textcolor{myblack}{Lê Hoàng Anh Vũ} & \textcolor{myblack}{2115319} \\
\hline
\end{tabular}
\end{table}



\end{titlepage}


%\thispagestyle{empty}

\newpage
\doublespacing
\tableofcontents
\onehalfspacing
\newpage
\section{TASK 1: REQUIREMENT ELICITATION}
\subsection{Describe the domain context of a smart printing service for students at HCMUT. Who are relevant stakeholders? What are their current needs? In your opinion, what benefits HCMUT\_SSPS will be for each stakeholder?}
\subsubsection{Bối cảnh của dự án}
Trong thời đại ngày này, giáo dục là quốc sách hàng đầu cho sự phát triển của đất nước. Trong đó, để có thể học tập một cách hiệu quả, thì các tài liệu luôn là những nhân tố không thể thiếu đối với mỗi học sinh, sinh viên. Vì vậy, nhu cầu in tài liệu để học tập đang ngày một tăng lên. Trong khuôn viên nhà trường, với số lượng sinh viên lớn, để có thể  đáp ứng nhu cầu in tài liệu học tập cũng như tiết kiệm thời gian quý báu cho các sinh viên, hệ thống dịch vụ in ấn thông minh dành cho sinh viên tại Trường Đại học Bách Khoa TP.HCM (HCMUT\_SSPS) cần được ra đời nhằm đáp ứng những nhu cầu thiết thực này. \par
Chương trình dịch vụ in ấn thông minh dành cho sinh viên tại Trường Đại học Bách Khoa TP.HCM (HCMUT\_SSPS) là một hệ thống cho phép sinh viên in tài liệu một cách thuận tiện và hiệu quả. Hệ thống bao gồm các máy in xung quanh khuôn viên trường và ứng dụng web hoặc ứng dụng di động để sử dụng các dịch vụ tiện ích liên quan của hệ thống. Sinh viên có thể tải lên tệp tài liệu lên hệ thống, chọn máy in và chỉ định các thuộc tính in. Phòng quản lí dịch vụ in ấn sinh viên (Student Printing Service Officer SPSO), cũng đồng thời có thể quản lí các máy in, kiểm tra lịch sử hoạt động, xuất báo cáo và nhiều các chức năng khác. Ngoài ra, để tăng tính bảo mật, cũng như đảm bảo quyền lợi cho sinh viên Bách Khoa, tài khoản muốn in tài liệu cần phải được xác thực bằng dịch vụ xác thực HCMUT\_SSO, tức là chỉ sinh viên Bách Khoa mới có quyền sử dụng dịch vụ in ấn tại khuôn viên nhà trường, tránh tình trạng người ngoài tiện đường vào trường in ấn gây mất thời gian cho sinh viên, cũng như tăng tính bảo mật cho hệ thống. 
\subsubsection{Các bên liên quan (stakeholders)}
Các bên liên quan có liên quan của hệ thống và những điều họ cần bao gồm:\par
 Sinh viên: Sinh viên cần một dịch vụ in ấn thuận tiện và hiệu quả trong khuôn viên trường. Đồng thời cần có thể in tài liệu của mình một cách nhanh chóng và dễ dàng, cũng như cần có thể theo dõi lịch sử in ấn của mình.Tài liệu khi được in ra cần được đảm bảo chính xác nhất có thể. In ấn an toàn và bảo mật thông tin, đảm bảo tài liệu được bảo vệ. Ngoài ra, sinh viên cần một hệ thống ít lỗi, dễ thao tác, thân thiện với người dùng. \par
 Cán bộ Quản lý Dịch vụ In ấn Sinh viên (SPSO): SPSO chịu trách nhiệm quản lý dịch vụ in ấn và đảm bảo nó hoạt động trơn tru. Họ cần các công cụ để giám sát và kiểm soát khả dụng máy in, theo dõi lịch sử in ấn và cấu hình cài đặt hệ thống.\par
Đơn vị xác thực HCMUT\_SSO (hệ thống xác thực sinh viên Bách Khoa): Để đảm bảo quyền lợi, tính bảo mật cho sinh viên Bách Khoa, hệ thống xác thực cần phải đảm bảo rằng chỉ khi có tài khoản được xác thực trong nội bộ Trường Đại học Bách Khoa TP.HCM mới được sử dụng phần mềm. Do đó, hệ thống xác thực cần biết được MSSV, họ và tên sinh viên khi sử dụng phần mềm. \par
Đơn vị quản lý BKPay: Sinh viên nếu cần in tài liệu nhiều hơn so với mức bình thường thì cần phải trả thêm 1 khoản qua BKPay để có thể tiếp tục in. BKPay cần biết được số lượng tài liệu cần được in thêm từ sinh viên để có thể cho ra hoá đơn.\par
Nhân viên trực máy in: Mỗi máy in phải cần có ít nhất một người trực để thực hiện các thao tác in ấn. Nhân viên trực máy in cần biết được tài liệu cần được in, số lượng bản,các trang in trong tài liệu, họ tên, mssv của sinh viên sử dụng dịch vụ để có thể thực hiện việc in một cách chính xác, nhanh chóng.
\subsubsection{Lợi ích của các bên liên quan tới hệ thống}
Theo nhóm em, một số lợi ích có thể có của các bên liên quan về hệ thống này: \\
\begin{table}[H]
\centering
\arrayrulecolor{black}
\begin{tabularx}{\textwidth}{| c | X |}
\hline
\multicolumn{1}{|c|}{\textbf{Các bên liên quan}} & \multicolumn{1}{c|}{\textbf{Lợi ích}}  \\
\hline
Sinh viên & 
\textbullet~In tài liệu một cách nhanh chóng và dễ dàng, không cần phải xếp hàng dài chờ đợi.  \\
& \textbullet~Sinh viên có thể theo dõi lịch sử in ấn của mình để quản lý chi phí.  \\
& 
\textbullet~In tài liệu từ bất kỳ máy in nào trong khuôn viên trường, thuận tiện cho giờ học cũng như vị trí học.  \\
& \textbullet~Sinh viên không cần lo lắng về tính bảo mật của tài khoản.   \\ 
\hline
SPSO & 
\textbullet~Dễ dàng giám sát và kiểm soát khả dụng máy in.  \\
& \textbullet~Có thể tạo báo cáo về việc sử dụng in ấn và chi phí để hỗ trợ ra quyết định.  \\
& 
\textbullet~Quản lý các cài đặt hệ thống một cách dễ dàng và tiện lợi.  \\
& \textbullet~Dễ dàng nâng cấp hay bảo trì hệ thống.   \\ 
\hline
Đơn vị quản lý BKPay & 
\textbullet~Dễ dàng hơn trong việc tính toán cho ra hoá đơn.  \\
& \textbullet~Tăng thêm 1 lượng thu nhập từ việc in thêm của học sinh.  \\ 
\hline
Đơn vị xác thực SSO  & 
\textbullet~Dễ dàng quản lý được ai là người in tài liệu.\\ 

\hline
Nhân viên trực máy in & 
\textbullet~Được nhận thêm công việc có lương, môi trường làm việc lành mạnh.  \\
& \textbullet~Thực hiện việc in tài liệu cho sinh viên/giảng viên giờ đây đã dễ dàng và chính xác hơn nhiều so với trước khi có hệ thống.   \\ 
& 
\textbullet~Tiếp kiệm được thời gian hơn cho người trực máy in. \\ 
\hline
\end{tabularx}

\end{table}
\subsection{Describe all functional and non-functional requirements that can be inferred from the project description.}
\begin{xltabular}{\textwidth}{| c | X | X |}
\hline
\multicolumn{1}{|c|}{\textbf{Các bên liên quan}} & \multicolumn{1}{c|}{\textbf{Yêu cầu chức năng}} & \multicolumn{1}{c|}{\textbf{Yêu cầu phi chức năng}} \\
\hline
Sinh viên & 
\textbullet~Tải lên và chọn tệp cần in qua trang web hoặc ứng dụng di động. & \textbullet~Giao diện dễ sử dụng. \\
& \textbullet~Lựa chọn máy in và tuỳ chỉnh các thuộc tính in khác như khổ giấy, số trang, in một mặt/hai mặt, số bản sao. &  \textbullet~Xử lý nhanh chóng và ổn định khi có nhiều người dùng cùng lúc (mỗi sinh viên không phải đợi quá 15 phút để được thực hiện in). \\
& 
\textbullet~Kiểm tra lịch sử in bao gồm thông tin sinh viên, máy in, tên tệp, thời gian bắt đầu và kết thúc in, số trang cho từng khổ giấy. & \\
& 
\textbullet~Thanh toán qua BKPAY. &  \\
\hline
SPSO & 
\textbullet~Quản lý máy in: thêm máy in, bật/tắt máy in trong hệ thống. & \textbullet~Tính năng báo cáo phải sẵn có 24/7 và cung cấp thông tin rõ ràng và ngắn gọn để hỗ trợ ra quyết định. \\
& \textbullet~Tuỳ chỉnh cấu hình hệ thống: cài đặt loại tệp được phép và số lượng trang mặc định cho sinh viên trong mỗi học kỳ. &  \textbullet~Báo cáo hệ thống phải đảm bảo độ chính xác của dữ liệu và có khả năng xuất báo cáo ở định dạng chuẩn như PDF, Excel. \\
& 
\textbullet~Xem lịch sử in ấn (nhật ký): xem lịch sử in của tất cả học sinh hoặc một học sinh trong một khoảng thời gian (từ ngày đến hiện tại) và cho phép lựa chọn máy in. & \textbullet~Hệ thống phải được vận hành trơn tru, nhất là những lúc cao điểm (nhiều sinh viên cần in tài liệu).\\
& 
\textbullet~Xem báo cáo: xem báo cáo thống kê sử dụng tài nguyên, mức tiêu thụ giấy và chi phí liên quan. &  \\
& 
\textbullet~Quản lý vận hành hệ thống. &  \\
\hline
Đơn vị quản lý BKPay& 
\textbullet~Tính hoá đơn, in, thanh toán hoá đơn. & \textbullet~Thời gian giao dịch nhanh chóng (phản hồi không quá 5 phút). \\
& \textbullet~Xem lịch sử giao dịch, thông báo giao dịch thành công. &  \textbullet~Giao dịch đảm bảo chính xác. \\
& 
\textbullet~Trợ giúp, chăm sóc khách hàng khi có sự cố. & \\
\hline
Đơn vị xác thực SSO & 
\textbullet~Xác thực người dùng: Hệ thống phải có khả năng xác thực người dùng và kiểm soát quyền truy cập của họ. & \textbullet~Hiệu suất cao và phản hồi nhanh: Tích hợp SSO phải đảm bảo hiệu suất và phản hồi nhanh để giảm độ trễ đăng nhập và nâng cao trải nghiệm người dùng (độ trễ dưới 1 giây, góp phần cho hệ thống hoạt động mượt mà). \\
& \textbullet~Tích hợp và đồng bộ hoá tài khoản: Tích hợp với SSPS để đồng bộ hoá tài khoản sinh viên với tài khoản dịch vụ. &  \textbullet~Bảo mật dữ liệu: Việc tích hợp SSO với SSPS phải tuân thủ các biện pháp bảo mật nghiêm ngặt để đảm bảo tính bảo mật và sẵn có của dữ liệu xác thực và thông tin nhạy cảm. \\
& 
\textbullet~Kiểm soát vai trò người dùng: Quản lý vai trò người dùng (sinh viên, giảng viên, quản trị hệ thống/người dùng khác) và cấp quyền truy cập phù hợp. & \textbullet~Khả năng mở rộng: Hệ thống SSO phải được thiết kế để xử lý số lượng người dùng và khả năng tải hệ thống tăng lên, đảm bảo hiệu suất không giảm (Trường Đại học Bách Khoa có khoảng 19.000 sinh viên chính quy, 4.000 học viên cao học, nghiên cứu sinh, có gần 1.000 giảng viên và sẽ tăng nhanh hơn nữa).\\
& 
\textbullet~Ghi nhật ký và kiểm tra: Ghi nhật ký và kiểm tra các hoạt động xác thực và truy cập của người dùng trong SSPS để hỗ trợ giám sát và khắc phục sự cố bảo mật. & \textbullet~Tích hợp với hệ thống khác: Tích hợp SSO phải tương thích và cho phép trao đổi dữ liệu với các hệ thống và dịch vụ khác của trường đại học. \\

\hline
Nhân viên trực máy in & 
\textbullet~Xem thông tin tài liệu, các trang cần thực hiện việc in ấn. & \textbullet~Tiếp nhận yêu cầu nhanh (nhân viên cần phải thực hiện việc in ấn nghiêm túc ngay sau khi nhận được yêu cầu, thường không quá 5 phút). \\
& \textbullet~Thông tin về sinh viên: tên sinh viên, MSSV của sinh viên yêu cầu dịch vụ. &  \textbullet~Đảm bảo việc in tài liệu thực hiện một cách chính xác, đúng máy (nếu có lỗi phải trợ giúp khách hàng). \\
& 
\textbullet~Xem vị trí máy in sinh viên yêu cầu in tài liệu. & \textbullet~Đảm bảo việc in chính xác, đúng tài liệu và các trang, số lượng mà sinh viên yêu cầu.\\
& 
\textbullet~Hiển thị thông tin tài liệu đã được in xong. & \textbullet~Đảm bảo chuyển tài liệu đã in cho đúng người yêu cầu, đúng vị trí mà sinh viên mong muốn (đảm bảo chính xác, hiệu suất cao, không bắt sinh viên chờ lâu quá 15 phút).\\
&  & \textbullet~Cần sẵn tại phòng máy đủ số lượng giấy a4, a3,...và mực in và thay mực khi cần (9.000 sinh viên chính quy, 4.000 học viên cao học, nghiên cứu sinh), cần ước tính được số lượng giấy và mực mỗi ngày để đáp ứng nhu cầu khách hàng.\\


\hline
\end{xltabular}
\subsection{Draw a use-case diagram for the whole system.
Choose at least one important module and draw its use-case diagram, as 
well as describe the use-cases using a table format}

\begin{figure}[H]
  \centering
  \vspace{0.5 cm}
  \includegraphics [scale=0.35] {data/1_3all.png}
  \vspace{1 cm}
  \caption{1.3 All USE CASE DIAGRAM }
\end{figure}
\subsubsection{Quản lý máy in}
\begin{figure}[H]
    \centering
    \includegraphics[width = 1\textwidth]{Images/use-case-diagram/manage_printers.png}
    \newline
    \newline
    
    \caption{Use-case diagram của nhánh chức năng quản lý máy in của SPSO}
    \label{fig:enter-label}
\end{figure}
\subsubsubsection{Xem thông tin máy in}
\begin{xltabular}{\textwidth}{|c|X|}
    \hline
    \textbf{Tên use-case} & \textbf{Xem thông tin máy in} \\
    \hline
    Actor & SPSO \\
    \hline
    Descriptions & Các SPSO sử dụng chức năng này để xem thông tin máy in. \\
    \hline 
    Precondition &Có danh sách các máy in hiện có trong hệ thống. 
    \newline SPSO đang ở trang "Quản lý"
    \\
    \hline
    Postcondition &Hệ thống hiển thị thông tin của máy in .\\
    \hline
    Trigger &SPSO chọn vào nút "Quản lý" ở trên thanh điều hướng.\\
    \hline
    Normal Flows & 
    1. Hệ thống mở cửa sổ quản lý máy in.
    \newline
    2. Hệ thống lấy danh sách các máy in.
    \newline
    3. Hệ thống hiển thị danh sách các máy in lên màn hình.
    \newline
    4. SPSO chọn vào máy in muốn xem thông tin.
    \newline
    5. Hệ thống lấy thông tin chi tiết của máy in và hiển thị cho SPSO.
    \newline
    \\
    \hline
    Exception Flows & None
    \\
    \hline
    Alternative Flows &
    - Tìm kiếm máy in bằng id sau bước 3.
    \newline
    - \textit{Extended points}: Bật/tắt máy in (tại bước 5)
    \\
    \hline
\end{xltabular}
\subsubsubsection{Thêm máy in}
\begin{xltabular}{\textwidth}{|c|X|}
    \hline
    \textbf{Tên use-case} & \textbf{Thêm máy in} \\
    \hline
    Actor & SPSO \\
    \hline
    Descriptions & Các SPSO sử dụng chức năng này để thêm máy in mới. \\
    \hline 
    Precondition &SPSO đang ở trang "Quản lý". \\
    \hline
    Postcondition &Hệ thống cập nhật máy in mới vào database .\\
    \hline
    Trigger &SPSO chọn vào nút "Thêm máy in" ở trong giao diện "Quản lý".
    \\
    \hline
    Normal Flows & 
    1. Hệ thống hiển thị trang để SPSO nhập thông tin máy in.
    \newline
    2. SPSO nhập các thông tin của máy in và ấn vào nút "Thêm".
    \newline
    3. Hệ thống cập nhật thông tin của máy in vừa được thêm vào.
    \newline
    4. Hệ thống hiển thị thông báo thêm máy in thành công.
    \newline
    \\
    \hline
    Exception Flows & Ở bước 2:\newline
    2a. SPSO nhập vào thông tin của một máy in đã có trong hệ thống.\newline
    2b. Hệ thống sẽ hiển thị thông báo máy in này đã có trong hệ thống và yêu cầu SPSO nhập lại thông tin.
    \\
    \hline
    Alternative Flows & None
    
    \hline
\end{xltabular}
\subsubsubsection{Bật/tắt máy in}
\begin{xltabular}{\textwidth}{|c|X|}
    \hline
    \textbf{Tên use-case} & \textbf{Bật/tắt máy in} \\
    \hline
    Actor & SPSO \\
    \hline
    Descriptions & Các SPSO sử dụng chức năng này để bật/tắt các máy in đang được quản lý trong hệ thống. \\
    \hline 
    Precondition &Có danh sách các máy in hiện có trong hệ thống. 
    \newline
    SPSO đang ở trang "Quản lý"
    \\
    
    \hline
    Postcondition &Máy in được chọn hiển thị trạng thái "bật" hoặc "tắt" do SPSO chọn.\\
    \hline
    Trigger &SPSO chọn vào nút "Bật/tắt" ở trong phần thông tin của máy in ngay ở giao diện hiển thị danh sách các máy in hoặc ở trong giao diện hiển thị thông tin chi tiết của một máy in cụ thể.\\
    \hline
    Normal Flows & 
    1. SPSO chọn máy in muốn thao tác.
    \newline
    2. Hệ thống hiển thị thông tin của máy in đó.
    \newline
    3. SPSO chọn vào nút "bật/tắt" để chuyển đổi qua lại giữa hai trạng thái của máy in.
    \newline
    4. Hệ thống lưu lại trạng thái của máy in.
    \\
    \hline
    Exception Flows & None
    \\
    \hline
    Alternative Flows & None
    \\
    \hline
\end{xltabular}


\subsubsection{In tài liệu}
\begin{figure}[H]
    \centering
    \includegraphics[width = 1\textwidth]{Images/use-case-diagram/printing.png}
    \caption{Use-case diagram của nhánh chức năng in tài liệu của sinh viên}
    \label{fig:enter-label}
\end{figure}
\begin{xltabular}{\textwidth}{|c|X|}
    \hline
    \textbf{Tên use-case} & \textbf{In tài liệu} \\
    \hline
    Actor & Sinh viên \\
    \hline
    Descriptions & Sinh viên sử dụng chức năng này để in tài liệu. \\
    \hline 
    Precondition & Sinh viên muốn in tài liệu trước hết phải đăng nhập, tải tài liệu lên, chọn máy in và các thuộc tính in. \\
    \hline
    Postcondition & Sinh viên in thành công và trở về màn hình chính.\\
    \hline
    Normal Flows & 
    1. Nhấn chọn "In tài liệu"
    \newline
    2. Hệ thống sẽ chuyển hướng trang in tài liệu.
    \newline
    3. Tại giao diện này, sinh viên sẽ tiến hành thực hiện 3 bước : 
    \newline
    4. Sinh viện chọn tài liệu cần tải lên và nhấn nút "Tải tài liệu lên".
    \newline
    5. Sinh viên chọn loại máy in để thực hiện in.
    \newline
    6. Sinh viên chọn các thuộc tính như cỡ giấy, số trang cần in, format, ...để in.
    \newline
    7. Sinh viên bấm nút "In".
    \newline
    8. Hệ thống sẽ tiến hành in và thông báo kết quả cho sinh viên.
    \\
    \hline
    Exception Flows & 
    Ở bước 4:
    \newline
    4a. Sinh viên tải sai định dạng của tài liệu 
    \newline
    4b. Hệ thống thông báo lỗi và yêu cầu sinh viên tải lại tài liệu khác.
    \newline
    Ở bước 6:
    \newline
    6a. Sinh viên không đủ số lượng giấy hiện có trong tài khoản để in.
    \newline
    6b. Hệ thống thông báo lỗi và chuyển hướng qua BKPay để sinh viên thực hiện mua thêm giấy in. \\
    \hline
    Alternative Flows & 
    Ở bước 5 và bước 6, nếu sinh viên không thực hiện các bước này thì hệ thống sẽ tự động sử dụng các option mặc định và chuyển sang bước 7. \\
    \hline
\end{xltabular}

\subsubsection{Tải tài liệu}
\begin{figure}[H]
    \centering
    \includegraphics[width = 1\textwidth]{Images/use-case-diagram/uploadfile.png}|
    \vspace{0.1cm}
    \caption{Use-case diagram của nhánh chức năng tải tệp tài liệu lên hệ thống cho sinh viên}
    \label{fig:enter-label}
\end{figure}
\begin{xltabular}{\textwidth}{|c|X|}
    \hline
    \textbf{Tên use-case} & \textbf{Tải tài liệu} \\
    \hline
    Actor & Sinh viên \\
    \hline
    Descriptions & Sinh viên sử dụng chức năng này nhằm tải tệp tài liệu lên hệ thống để chuẩn bị in ấn. \\
    \hline 
    Precondition & Sinh viên muốn tải tệp tài liệu lên hệ thống trước hết phải xác thực và đăng nhập vào hệ thống. \\
    \hline
    Postcondition & Tệp tài liệu đã được tải lên hệ thống và sẵn sàng để in.\\
    \hline
    Normal Flows & 
    1. Nhấn chọn "Tải tài liệu."
    \newline
    2. Hệ thống hiển thị giao diện cho phép sinh viên chọn tệp tài liệu từ máy tính hoặc thiết bị di động của họ.
    \newline
    3. Sinh viên tìm và chọn tệp tài liệu mà họ muốn in.
     \newline
    4. Sinh viên nhấn chọn "Tải lên" để bắt đầu quá trình tải tệp lên.
    \newline
    5. Hệ thống xử lý và lưu trữ tệp tài liệu trên hệ thống.
    \newline
    6. Hệ thống gửi thông báo xác nhận thành công cho sinh viên và hiển thị tệp tải lên trong danh sách tài liệu của họ.\\
    \hline
    Exception Flows & 
    None.\\
    \hline
    Alternative Flows & None.\\
    \hline
\end{xltabular}
\newpage

\section{TASK 2: SYSTEM MODELING}
\subsection{Activity Diagram}
\begin{figure}[H]
  \centering
  \vspace{0.5 cm}
  \includegraphics [scale=0.45] {Images/ActivityDiagram4.0.png}
  \vspace{1 cm}
  \caption{Activity Diagram }
\end{figure}
\newpage
\begin{itemize}
\item Sinh viên truy cập vào trang chủ của hệ thống và bắt đầu đăng nhập. \par
\item Sau khi đăng nhập thành công, sinh viên có thể thực hiện upload file cần in ấn lên hệ thống.  \par
\item SPSO sẽ kiểm tra xem file mà sinh viên gửi có format hợp lệ không.\par    
\begin{itemize}
\item Nếu hợp lệ, sinh viên được đi tới bước tiếp theo và được phép chọn thuộc tính về in ấn (số tờ, các mặt, loại tờ,...)
\item Nếu không hợp lệ, sinh viên cần phải upload lại file cần in. \par

\end{itemize}

\item Sau khi sinh viên chọn các thuộc tính, SPSO sẽ check lại số lượng giấy mà sinh viên được in không cần trả thêm. \par   

\begin{itemize}
\item Nếu số lượng giấy sinh viên cần in chưa vượt quá số lượng được in không trả thêm còn lại, SPSO cho phép sinh viên được chọn máy in mà sinh viên đó mong muốn in, đồng thời sẽ trừ đi lượng giấy in không cần trả thêm của sinh viên. 
\item Trường hợp còn lại, SPSO sẽ gửi số lượng lên BKPay để BKay thực hiện nhiệm vụ tính toán lượng cần trả thêm. Sau đó BKPay sẽ gửi cho sinh viên hóa đơn. Sinh viên trả tiền sẽ được BKPay xác nhận giao dịch và đi tới bước chọn máy in.\par

\end{itemize}

\item Sau khi sinh viên chọn máy in mong muốn được in tài liệu, SPSO sẽ kiểm tra tình trạng của máy in này. 
\begin{itemize}
\item Nếu máy in có thể hoạt động tốt, SPSO sẽ thêm vào lịch sử các tài liệu cần in. 
\item Trường hợp còn lại khi máy in sinh viên yêu cầu có vấn đề, SPSO sẽ yêu cầu sinh viên chọn lại máy in.\par

\end{itemize}
\item Sau khi SPSO update lại history, nhân viên trực máy sẽ nhận được thông báo và căn cứ theo history này để thực hiện việc in tài liệu cho sinh viên.  \par
\item Cuối cùng, sinh viên tới máy in mà sinh viên mong muốn in để nhận tài liệu và rời hệ thống.\par

\end{itemize}




\newpage
\subsection{Sequence Diagram}
\subsubsection{In tài liệu}
\begin{figure}[H]
    \centering
    \includegraphics[scale=0.65]{Images/sequence-diagram/printing.png}
    \vspace{0.1cm}
    \caption{Sequence diagram của nhánh chức năng in tài liệu}
\end{figure}
\begin{itemize}
\item Sinh viên sẽ bắt đầu thực hiện in tài liệu
\item Hệ thống sẽ tiếp nhận và chuyển đến trang in tài liệu.
\item Sinh viên sau đó sẽ tiến hành các bước sau đây:
\begin{itemize}
\item Sinh viên tiến hành tải tài liệu cần in lên hệ thống, hệ thống sẽ kiểm tra định dạng của tài liệu. Nếu tài liệu có định dạng sai so với quy định thì hệ thống sẽ thông báo lỗi và yêu cầu sinh viên tải lại tài liệu khác.
\item Sau đó, sinh viên tiến hành chọn máy in. Hệ thống sẽ hiển thị danh sách thông tin các máy in sẵn có và sinh viên sẽ chọn máy in phù hợp để in tài liệu.
\item Cuối cùng, sinh viên sẽ điều chỉnh các thuộc tính in cho phù hợp với yêu cầu của mình. Trong trường hợp sinh viên không đủ số lượng giấy hiện có trong tài khoản, hệ thống sẽ thông báo lỗi và chuyển hướng qua BKPay để sinh viên thực hiện mua thêm giấy in.
\end{itemize}
\item Sau khi hoàn thành các bước trên, sinh viên sẽ tiến hành in tài liệu.
\item Hệ thống sẽ thông báo kết quả và lưu vào lịch sử in.
\end{itemize}

\subsubsection{Bật, tắt in}
\begin{figure}[H]
    \centering
    \includegraphics[width = 1\textwidth]{Images/sequence-diagram/enable_disable_printer.png}
    \newline
    \newline
    \caption{Sequence diagram của nhánh chức năng bật/tắt máy in}
    \label{fig:enter-label}
\end{figure}
\textbf{Mô tả}
\begin{itemize}
    \item Trong chức năng này SPSO sẽ nhập ID để tìm máy in muốn thao tác
    \item Sau khi nhận được ID, hệ thống kiểm tra ID của máy in trong Database. Sẽ có hai trường hợp:
    \begin{itemize}
        \item Nếu tìm thấy ID
        \begin{itemize}
            \item SPSO sẽ gửi yêu cầu bật/tắt máy in
            \item Hệ thống tìm tới máy in SPSO muốn thao tác và thực hiện yêu cầu bật/tắt
        \end{itemize}
        \item Nếu không tìm thấy ID hệ thống sẽ gửi thông báo "Không tìm thấy máy in" cho SPSO
    \end{itemize}
\end{itemize}

\subsubsection{Thêm máy in}
\begin{figure}[H]
    \centering
    \includegraphics[width = 1\textwidth]{Images/sequence-diagram/add_printer.png}
    \newline
    \newline
    \caption{Sequence diagram của nhánh chức năng thêm máy in}
    \label{fig:enter-label}
\end{figure}

\textbf{Mô tả}
\begin{itemize}
    \item Trong chức năng này SPSO sẽ nhập thông tin của máy in muốn thêm vào database
    \item Hệ thống kiểm tra ID của máy in trong Database. Sẽ có hai trường hợp:
    \begin{itemize}
        \item Nếu không tìm thấy ID, SPSO sẽ gửi yêu cầu thêm máy in mới vào database
        \item Nếu tìm thấy ID hệ thống sẽ gửi thông báo "Máy in đã có trong database" cho SPSO
    \end{itemize}
\end{itemize}

\subsubsection{Tải tài liệu}
\begin{figure}[H]
    \centering
    \includegraphics[scale=0.7]{Images/sequence-diagram/uploadfile_diagram.png}
    \newline
    \newline
    \caption{Sequence diagram của nhánh chức năng tải tài liệu}
    \label{fig:enter-label}
\end{figure}

\textbf{Mô tả}
\begin{itemize}
\item Người dùng sẽ tải tệp tài liệu lên.
\item Hệ thống in thông minh hiển thị giao diện cho phép sinh viên chọn tệp tài liệu từ máy tính hoặc thiết bị di động của họ.
\item  Người dùng chỉ định các thuộc tính in như kích thước giấy, số trang cần in, in một mặt/ hai mặt, số lượng bản,v.v.. và chọn tải lên để bắt đầu quá trình tải tệp lên.
\item Hệ thống xử lý yêu cầu của người dùng và tạo một tệp tạm để lưu trữ tài liệu tải lên. Sau đó hệ thống gửi yêu cầu tải lên tới máy chủ (Server) để lưu trữ tệp tài liệu.
\item Máy chủ tiếp nhận yêu cầu tải lên và lưu trữ tệp tài liệu trong hệ thống.
\item Máy chủ gửi thông báo xác nhận thành công cho HCMUT\_SSPS.
\item Hệ thống in thông minh nhận thông báo xác nhận và hiển thị tệp tải lên trong danh sách tài liệu.
\item Người dùng nhận thông báo xác nhận và có thể xem tệp đã tải lên.
\end{itemize}

\subsection{Class Diagram}
\subsubsection{In tài liệu}
\begin{figure}[H]
    \centering
    \includegraphics[scale=0.68]{Images/class-diagram/printing.png}
    \vspace{0.1cm}
    \caption{Class diagram của nhánh chức năng in tài liệu}
\end{figure}
\begin{itemize}
    \item Sau khi chọn in tài liệu thì ở phần PrintPage, sinh viên có thể thao tác Upload(tải tài liệu), ChoosePrinter(chọn máy in) và ChangeProp(chỉnh sửa các thuộc tính in)
    \item PrintController sẽ là nơi xử lí các yêu cầu logic bao gồm RequestPrinterInfo(yêu cầu thông tin máy in), CheckFileType(kiểm tra định dạng file) và CheckProp(kiểm tra các thuộc tính in)
    \item Ở phần PrintModel sẽ lưu trữ ID, danh sách các file tài liệu, danh sách các máy in (được lấy từ method GetPrinterInfo của Database). Đồng thời có thể cập nhật các danh sách tài liệu, máy in, ... thông qua PrintModel.
    \item Từ PrintModel chúng ta sẽ sinh ra 2 class con là printerModel và fileModel:
    \begin{itemize}
        \item printerModel sẽ lưu trữ thông tin của các máy in bao gồm ID, nhãn hiệu, địa điểm, thông tin và trạng thái của máy in. Ở đây chúng ta có thể thêm, lấy hoặc set trạng thái cho các máy in.
        \item fileModel sẽ lưu trữ thông tin của các file tài liệu được tải lên bao gồm ID, định dạng tài liệu, tên, số trang, đã được kiểm tra định dạng hay chưa. Đồng thời chúng ta cũng có thể thêm, lấy hoặc set trạng thái cho các tài liệu này.
    \end{itemize}
    \item Trong trường hợp sinh viên hết số lượng giấy trong tài khoản, hệ thống sẽ chuyển hướng sang BKPay và tại đây sinh viên có thể thực hiện mua thêm giấy và sử dụng method MakePayment để thanh toán trên BKPay.
    \item Sau khi thực hiện hết tất cả các bước trên, PrintController sẽ gửi yêu cầu đến Database để thay đổi trạng thái in PrintingStatus và đưa ra thông báo trên màn hình.
\end{itemize}

\subsubsection{Quản lý máy in}
\begin{figure}[H]
    \centering
    \includegraphics[width = 1\textwidth]{Images/class-diagram/manage_printer.png}
    \newline
    \newline
    \caption{Class diagram của nhánh chức năng bật/tắt/thêm máy in}
    \label{fig:enter-label}
\end{figure}
\begin{itemize}
    \item \textit{SPSO} thực hiện các chức năng thêm/bật/tắt máy in thông qua giao diện "Quản lý máy in".
    \item \textit{PrinterManagementSystem} là nơi lưu trữ danh sách các máy in có trong hệ thống.
    \item Với nhánh chức năng bật/tắt máy in, SPSO sẽ chọn trực tiếp máy in đang hiển thị trên giao diện hoặc tìm kiếm thông qua ID, sau đó gửi yêu cầu tới hệ thống. Hệ thống sẽ tiếp nhận các yêu cầu và thực hiện các chức năng \textit{enablePrinter()}, \textit{disablePrinter()} trên máy in đã được chọn
    \item Với nhánh chức năng thêm máy in, sau khi nhận được yêu cầu từ \textit{SPSO} \textit{PrinterManagementSystem} sẽ tìm ID của máy in mới được yêu cầu thêm vào, nếu đã tồn tại, hệ thống sẽ gửi thông báo cho SPSO biết là đã có máy in này trong hệ thống, nếu không tìm thấy ID hệ thống sẽ thực hiện chức năng \textit{addPrinter()}
\end{itemize}
\newpage
\subsubsection{Tải tài liệu}
\begin{figure}[H]
    \centering
    \includegraphics[width = 1\textwidth]{Images/class-diagram/Upload.png}
    \newline
    \newline
    \caption{Class diagram của nhánh chức năng tải lên}
    \label{fig:enter-label}
\end{figure}

\begin{itemize}
    \item Sau khi chọn in tài liệu thì ở phần PrintPage, sinh viên có thể thao tác Upload(tải tài liệu)
    \item PrintController sẽ là nơi xử lí yêu cầu logic CheckFileType(kiểm tra định dạng file)
    \item Ở phần PrintModel sẽ lưu trữ ID, danh sách các file tài liệu. Đồng thời có thể cập nhật các danh sách tài liệu thông qua PrintModel.

    \item Sau khi thực hiện hết tất cả các bước trên, PrintController sẽ gửi yêu cầu đến Database để thay đổi trạng thái tải lên UploadingStatus và đưa ra thông báo trên màn hình.
\end{itemize}

\subsection{Develop MVP 1 as user interfaces of either a Desktop-view central 
dashboard for a particular module (the same with the module used in task 
2.1). Decide yourself what to include in the view. Use a wireframe tool like 
Figma or Adobe XD, or Illustrator}
\subsubsection{Default page}
\begin{figure}[H]
    \centering
    \includegraphics[width = 1\textwidth]{2.4/default_page.png}
    \newline
    \newline
    \caption{Trang chủ trước khi đăng nhập}
    \label{fig:enter-label}
\end{figure}

\subsubsection{Homepage}
\begin{figure}[H]
    \centering
    \includegraphics[width = 1\textwidth]{2.4/Trang chu.png}
    \newline
    \newline
    \caption{Trang chủ}
    \label{fig:enter-label}
\end{figure}

\subsubsection{Sign up}
\begin{figure}[H]
    \centering
    \includegraphics[width = 1\textwidth]{2.4/Dang ki.png}
    \newline
    \newline
    \caption{Trang đăng ký}
    \label{fig:enter-label}
\end{figure}

\subsubsection{Log in}
\begin{figure}[H]
    \centering
    \includegraphics[width = 1\textwidth]{2.4/Dang nhap.png}
    \newline
    \newline
    \caption{Trang đăng nhập}
    \label{fig:enter-label}
\end{figure}

\subsubsection{Printing}
\begin{figure}[H]
    \centering
    \includegraphics[width = 1\textwidth]{2.4/In tai lieu.png}
    \newline
    \newline
    \caption{Trang in tài liệu}
    \label{fig:enter-label}
\end{figure}

\subsubsection{Printing history}
\begin{figure}[H]
    \centering
    \includegraphics[width = 1\textwidth]{2.4/Lich su in.png}
    \newline
    \newline
    \caption{Trang xem lịch sử in}
    \label{fig:enter-label}
\end{figure}

\subsubsection{Printer management}
\begin{figure}[H]
    \centering
    \includegraphics[width = 1\textwidth]{2.4/Quan li may in.png}
    \newline
    \newline
    \caption{Trang quản lý máy in}
    \label{fig:enter-label}
\end{figure}


\section{TASK 3: ARCHITECTURE DESIGN}
\subsection{Architecture}
\subsubsection{Architecture design}

\begin{figure}[H]
    \centering
\includegraphics[scale=0.6]{Images/architecture-diagram/box-line.png}
    \vspace{0.1cm}
    \par
    \caption{Box-line diagram}
\end{figure}
\begin{itemize}
\item Hệ thống sử dụng kiến trúc lớp (Layered architecture) để thiết kế.
\item Presentation Layer (Lớp trình bày): Đây là lớp giao diện người dùng, chịu trách nhiệm hiển thị thông tin cho người dùng và nhận các yêu cầu từ họ.\par    
\begin{itemize}
\item Student Interface: Chế độ mặc định khi truy cập vào trang chính của hệ thống. Dành cho sinh viên và cho phép họ thực hiện việc in ấn tài liệu khi có nhu cầu. 
\item SPSO Interface: Chế độ dành cho SPSO. Cho phép quản lý hệ thống thêm, bớt máy in; thực hiện duyệt các tài liệu được yêu cầu in.  \par
\item Printer Staff Interface: Chế độ dành cho nhân viên gác máy in. Cho phép nhân viên gác máy có thể thấy được lịch sử gồm các tài liệu đã qua kiểm duyệt cần được in qua đó để thực hiện việc in ấn cho sinh viên.  \par

\end{itemize}
\item Interface Controller: Là lớp chịu trách nhiệm điều phối các tương tác giữa các giao diện người dùng (UI) và các lớp logic kinh doanh (Business logic) trong ứng dụng.\par


\item Business Layer (Lớp logic kinh doanh): Lớp này chứa logic của ứng dụng. \par   

\begin{itemize}
\item Thực hiện việc in ấn tài liệu cho sinh viên. \par
\item Xem lại lịch sử các tài liệu.\par
\item Quản lý hệ thống.\par
\item Thực hiện thanh toán.\par

\end{itemize}
\item Database Layer (Lớp dữ liệu): Lớp này quản lý việc truy cập và tương tác với cơ sở dữ liệu.\par    
\begin{itemize}
\item Printer Database: Lưu trữ danh sách các máy in có trong khuôn viên trường. 
\item History Database: Lưu trữ lịch sử các tài liệu yêu cầu/được in của hệ thống. \par

\end{itemize}

\item Server Manager: Điều phối tương tác giữa Database và các tác vụ ở lớp Business.\par
\item External services/APIs (Các dịch vụ nằm ngoài phạm vi của ứng dụng chính): Là các giao diện được cung cấp bởi các bên thứ ba. Hệ thống liên kết với các External services thông qua:\par    
\begin{itemize}
\item BK\textunderscore Pay: Thông qua chức năng Payment của lớp Business, chúng ta sẽ tạm thời rời hệ thống hiện tại để đi tới giao diện của BK\textunderscore Pay. 
\item HCMUT\textunderscore SSO: Thông qua Authetication của Interface controller, hệ thống sẽ đi tới giao diện của HCMUT\textunderscore SSO. \par

\end{itemize}


\end{itemize}


\subsubsection{Deployment}
\begin{figure}[H]
    \centering
\includegraphics[scale=0.35]{Images/architecture-diagram/deploy.png}
    \vspace{0.1cm}
    \caption{Deployment diagram }
\end{figure}
\begin{itemize}
\item Sinh viên và SPSO kết nối với application server thông qua trình duyệt web sử dụng phương thức TCP/IP. \par
\item Application server sẽ cung cấp chức năng đăng nhập cho mọi đối tượng.  \par
\item Sau khi đăng nhập, Application server sẽ sử dụng dịch vụ xác thực do HCMUT SSO cung cấp bằng cách gọi API theo phương thức HTTP được cung cấp từ HCMUT SSO, kết quả trả về sẽ xác định đối tượng đăng nhập là  sinh viên hay SPSO để cung cấp giao diện và dịch vụ tương ứng.\par    


\item Application server cung cấp giao diện và dịch vụ in ấn, xem lịch sử và thanh toán cho sinh viên sử dụng, quản lí cấu hình hệ thống, quản lí máy in, xem toàn bộ lịch sử cho SPSO. \par   


\item Database system sử dụng MongoDB atlas để lưu dữ liệu về máy in,số trang của mỗi sinh viên cũng như lịch sử in ấn. Application server giao tiếp với database system qua internet sử dụng MongoDB Wire Protocol và định tuyến bằng TCP/IP.  \par
\item Với các dịch vụ bên ngoài khác, được thiết kế để sử dụng trong artificial external service/ apis như trên hình. Sử dụng kết nối theo chuẩn được cung cấp từ nhà cung cấp dịch vụ. Ví dụ, đối với BKPay để thanh toán cho sinh viên.Application server có thể gửi yêu cầu HTTP đến hệ thống BKPay. Hệ thống BKPay sẽ trả lại URL thanh toán. Application server sau đó có thể chuyển hướng người dùng đến URL thanh toán để hoàn tất thanh toán.Để nhận thông báo khi thanh toán mua trang in được hoàn tất,Application server có thể đăng ký nhận webhooks được cung cấp bởi hệ thống BKPay. Khi thanh toán được hoàn tất, hệ thống BKPay sẽ gửi thông báo qua webhook. Từ đó có thể cập nhật số dư trang in của số trang cho phù hợp và cập nhật lại trong cơ sở dữ liệu.\par

\end{itemize}




\newpage
\subsection{Component Diagram}
\subsubsection{In tài liệu}
\begin{figure}[H]
    \centering
    \includegraphics[scale=0.38]{Images/component-diagram/printing.png}
    \vspace{0.1cm}
    \caption{Component diagram của nhánh chức năng in tài liệu}
\end{figure}
\begin{itemize}
    \item Hệ thống sẽ có 3 component lớn
    \begin{itemize}
        \item PrintPage component:
        \begin{itemize}
            \item Bao gồm Student Interface component
            \item Các component trong Student Interface đảm nhận nhiệm vụ như đã mô tả trong task 2.3
            \item Các component này yêu cầu interface từ PrintController để thực hiện lấy thông tin cần thiết.
        \end{itemize}
        \item PrintController component:
        \begin{itemize}
            \item Chứa các component Printer Controller, Properties và File Controller.
            \item Các component này đảm nhận các nhiệm vụ như đã mô tả trong task 2.3
            \item Properties và File Controller sẽ cung cấp interface cho Printer Controller để thực hiện việc kiểm tra định dạng file và các thuộc tính in.
            \item Ngoài ra PrintController sẽ cũng cấp các interface cho PrintPage để lấy thông tin và yêu cầu interface từ PrintModel
        \end{itemize}
        \item PrintModel component:
        \begin{itemize}
            \item Chứa các component Printer và File
            \item Các component này sẽ đảm nhận các nhiệm vụ như đã mô tả trong task 2.3
            \item PrintModel sẽ cung cấp các interface cho PrintController để lấy thông tin danh sách máy in và các file tài liệu được tải lên
        \end{itemize}
    \end{itemize}
    \item Ngoài ra còn có các component:
    \begin{itemize}
        \item BKPay component:
        \begin{itemize}
            \item Được dùng để thực hiện các thanh toán
            \item Cung cấp interface để PrintPage thực hiện thao tác với dữ liệu về payment
        \end{itemize}
        \item Database component:
        \begin{itemize}
            \item Thao tác trực tiếp với cơ sở dữ liệu
            \item Cung cấp các interface để PrintModel thực hiện truy xuất trực tiếp thông tin trên cơ sở dữ liệu
        \end{itemize}
    \end{itemize}
\end{itemize}

\subsubsection{Quản lý máy in}
\begin{figure}[H]
    \centering
    \includegraphics[width = 1\textwidth]{Images/component-diagram/component_diagram_manage_printer.png}
    \newline
    \newline
    \caption{Component diagram của nhánh chức năng quản lý máy in}
    \label{fig:enter-label}
\end{figure}

\textbf{\textit{Mô tả:}}
\begin{itemize}
    \item Trong giao diện của nhân viên dịch vụ in ấn (SPSO), dữ liệu về danh sách các máy in (\textit{Printer List}) có trong \textit{Printer Database} được tổng hợp bởi khối \textit{Get Printer Listing Info} và xuất ra giao diện thông qua khối \textit{Display Printer Listing}, tương tự dữ liệu chi tiết của một máy in cụ thể cũng được tổng hợp và xuất ra giao diện \textit{Display Printer Details}
    \item Thao tác bật/tắt máy in được xử lý thông qua khối \textit{Change Printer State}. Khối này nhận thông tin về trạng thái (bật/tắt) của máy in được SPSO chọn từ giao diện được xuất bởi khối \textit{Display Enable/Disable Printers}, sau đó cập nhật trạng thái đó vào máy in được chọn trong \textit{Printer Database}

    \item SPSO thao tác việc thêm máy in mới vào hệ thống thông qua giao diện được xuất bởi khối \textit{Display Add Printer}, cung cấp thông tin của máy in mới muốn thêm vào để khối \textit{Add new printer} xử lý. Khối \textit{Add new printer} sau khi nhận được thông tin thì tiến hành cập nhật máy in mới vào \textit{Printer List} trong component \textit{Printer Database}
\end{itemize}

\subsubsection{Tải tài liệu}
\begin{figure}[H]
    \centering
    \includegraphics[width = 1\textwidth]{Images/component-diagram/Uploadfile.png}
    \newline
    \newline
    \caption{Component diagram của nhánh chức năng tải tài liệu}
    \label{fig:enter-label}
\end{figure}
\begin{itemize}
    \item Hệ thống sẽ có 3 component lớn
    \begin{itemize}
        \item PrintPage component:
        \begin{itemize}
            \item Bao gồm Student Interface component
            \item Các component trong Student Interface đảm nhận nhiệm vụ như đã mô tả trong task 2.3
            \item Các component này yêu cầu interface từ PrintController để thực hiện lấy thông tin cần thiết.
        \end{itemize}
        \item PrintController component:
        \begin{itemize}
            \item Chứa component File Controller.
            \item Component này đảm nhận các nhiệm vụ như đã mô tả trong task 2.3
            \item File Controller sẽ cung cấp interface cho Printer Controller để thực hiện việc kiểm tra định dạng file.
            \item Ngoài ra PrintController sẽ cũng cấp các interface cho PrintPage để lấy thông tin và yêu cầu interface từ PrintModel
        \end{itemize}
        \item PrintModel component:
        \begin{itemize}
            \item Chứa component File
            \item Các component này sẽ đảm nhận các nhiệm vụ như đã mô tả trong task 2.3
            \item PrintModel sẽ cung cấp các interface cho PrintController để lấy thông tin danh sách các file tài liệu được tải lên
        \end{itemize}
    \end{itemize}
    \item Ngoài ra còn có Database component:
        \begin{itemize}
            \item Thao tác trực tiếp với cơ sở dữ liệu
            \item Cung cấp các interface để PrintModel thực hiện truy xuất trực tiếp thông tin trên cơ sở dữ liệu
        \end{itemize}
    \end{itemize}

\section{TASK 4: IMPLEMENTATION - SPRINT 1}
\newpage
\section{TASK 5: IMPLEMENTATION - SPRINT 2}
\newpage
\section{REFERENCES}

\end{document}