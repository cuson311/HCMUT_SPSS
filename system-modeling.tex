\section{TASK 2: SYSTEM MODELING}
\subsection{Activity Diagram}
\begin{figure}[H]
  \centering
  \vspace{0.5 cm}
  \includegraphics [scale=0.45] {Images/ActivityDiagram4.0.png}
  \vspace{1 cm}
  \caption{Activity Diagram }
\end{figure}
\newpage
\begin{itemize}
\item Sinh viên truy cập vào trang chủ của hệ thống và bắt đầu đăng nhập. \par
\item Sau khi đăng nhập thành công, sinh viên có thể thực hiện upload file cần in ấn lên hệ thống.  \par
\item SPSO sẽ kiểm tra xem file mà sinh viên gửi có format hợp lệ không.\par    
\begin{itemize}
\item Nếu hợp lệ, sinh viên được đi tới bước tiếp theo và được phép chọn thuộc tính về in ấn (số tờ, các mặt, loại tờ,...)
\item Nếu không hợp lệ, sinh viên cần phải upload lại file cần in. \par

\end{itemize}

\item Sau khi sinh viên chọn các thuộc tính, SPSO sẽ check lại số lượng giấy mà sinh viên được in không cần trả thêm. \par   

\begin{itemize}
\item Nếu số lượng giấy sinh viên cần in chưa vượt quá số lượng được in không trả thêm còn lại, SPSO cho phép sinh viên được chọn máy in mà sinh viên đó mong muốn in, đồng thời sẽ trừ đi lượng giấy in không cần trả thêm của sinh viên. 
\item Trường hợp còn lại, SPSO sẽ gửi số lượng lên BKPay để BKay thực hiện nhiệm vụ tính toán lượng cần trả thêm. Sau đó BKPay sẽ gửi cho sinh viên hóa đơn. Sinh viên trả tiền sẽ được BKPay xác nhận giao dịch và đi tới bước chọn máy in.\par

\end{itemize}

\item Sau khi sinh viên chọn máy in mong muốn được in tài liệu, SPSO sẽ kiểm tra tình trạng của máy in này. 
\begin{itemize}
\item Nếu máy in có thể hoạt động tốt, SPSO sẽ thêm vào lịch sử các tài liệu cần in. 
\item Trường hợp còn lại khi máy in sinh viên yêu cầu có vấn đề, SPSO sẽ yêu cầu sinh viên chọn lại máy in.\par

\end{itemize}
\item Sau khi SPSO update lại history, nhân viên trực máy sẽ nhận được thông báo và căn cứ theo history này để thực hiện việc in tài liệu cho sinh viên.  \par
\item Cuối cùng, sinh viên tới máy in mà sinh viên mong muốn in để nhận tài liệu và rời hệ thống.\par

\end{itemize}




\newpage
\subsection{Sequence Diagram}
\subsubsection{In tài liệu}
\begin{figure}[H]
    \centering
    \includegraphics[scale=0.65]{Images/sequence-diagram/printing.png}
    \vspace{0.1cm}
    \caption{Sequence diagram của nhánh chức năng in tài liệu}
\end{figure}
\begin{itemize}
\item Sinh viên sẽ bắt đầu thực hiện in tài liệu
\item Hệ thống sẽ tiếp nhận và chuyển đến trang in tài liệu.
\item Sinh viên sau đó sẽ tiến hành các bước sau đây:
\begin{itemize}
\item Sinh viên tiến hành tải tài liệu cần in lên hệ thống, hệ thống sẽ kiểm tra định dạng của tài liệu. Nếu tài liệu có định dạng sai so với quy định thì hệ thống sẽ thông báo lỗi và yêu cầu sinh viên tải lại tài liệu khác.
\item Sau đó, sinh viên tiến hành chọn máy in. Hệ thống sẽ hiển thị danh sách thông tin các máy in sẵn có và sinh viên sẽ chọn máy in phù hợp để in tài liệu.
\item Cuối cùng, sinh viên sẽ điều chỉnh các thuộc tính in cho phù hợp với yêu cầu của mình. Trong trường hợp sinh viên không đủ số lượng giấy hiện có trong tài khoản, hệ thống sẽ thông báo lỗi và chuyển hướng qua BKPay để sinh viên thực hiện mua thêm giấy in.
\end{itemize}
\item Sau khi hoàn thành các bước trên, sinh viên sẽ tiến hành in tài liệu.
\item Hệ thống sẽ thông báo kết quả và lưu vào lịch sử in.
\end{itemize}

\subsubsection{Bật, tắt in}
\begin{figure}[H]
    \centering
    \includegraphics[width = 1\textwidth]{Images/sequence-diagram/enable_disable_printer.png}
    \newline
    \newline
    \caption{Sequence diagram của nhánh chức năng bật/tắt máy in}
    \label{fig:enter-label}
\end{figure}
\textbf{Mô tả}
\begin{itemize}
    \item Trong chức năng này SPSO sẽ nhập ID để tìm máy in muốn thao tác
    \item Sau khi nhận được ID, hệ thống kiểm tra ID của máy in trong Database. Sẽ có hai trường hợp:
    \begin{itemize}
        \item Nếu tìm thấy ID
        \begin{itemize}
            \item SPSO sẽ gửi yêu cầu bật/tắt máy in
            \item Hệ thống tìm tới máy in SPSO muốn thao tác và thực hiện yêu cầu bật/tắt
        \end{itemize}
        \item Nếu không tìm thấy ID hệ thống sẽ gửi thông báo "Không tìm thấy máy in" cho SPSO
    \end{itemize}
\end{itemize}

\subsubsection{Thêm máy in}
\begin{figure}[H]
    \centering
    \includegraphics[width = 1\textwidth]{Images/sequence-diagram/add_printer.png}
    \newline
    \newline
    \caption{Sequence diagram của nhánh chức năng thêm máy in}
    \label{fig:enter-label}
\end{figure}

\textbf{Mô tả}
\begin{itemize}
    \item Trong chức năng này SPSO sẽ nhập thông tin của máy in muốn thêm vào database
    \item Hệ thống kiểm tra ID của máy in trong Database. Sẽ có hai trường hợp:
    \begin{itemize}
        \item Nếu không tìm thấy ID, SPSO sẽ gửi yêu cầu thêm máy in mới vào database
        \item Nếu tìm thấy ID hệ thống sẽ gửi thông báo "Máy in đã có trong database" cho SPSO
    \end{itemize}
\end{itemize}

\subsubsection{Tải tài liệu}
\begin{figure}[H]
    \centering
    \includegraphics[scale=0.7]{Images/sequence-diagram/uploadfile_diagram.png}
    \newline
    \newline
    \caption{Sequence diagram của nhánh chức năng tải tài liệu}
    \label{fig:enter-label}
\end{figure}

\textbf{Mô tả}
\begin{itemize}
\item Người dùng sẽ tải tệp tài liệu lên.
\item Hệ thống in thông minh hiển thị giao diện cho phép sinh viên chọn tệp tài liệu từ máy tính hoặc thiết bị di động của họ.
\item  Người dùng chỉ định các thuộc tính in như kích thước giấy, số trang cần in, in một mặt/ hai mặt, số lượng bản,v.v.. và chọn tải lên để bắt đầu quá trình tải tệp lên.
\item Hệ thống xử lý yêu cầu của người dùng và tạo một tệp tạm để lưu trữ tài liệu tải lên. Sau đó hệ thống gửi yêu cầu tải lên tới máy chủ (Server) để lưu trữ tệp tài liệu.
\item Máy chủ tiếp nhận yêu cầu tải lên và lưu trữ tệp tài liệu trong hệ thống.
\item Máy chủ gửi thông báo xác nhận thành công cho HCMUT\_SSPS.
\item Hệ thống in thông minh nhận thông báo xác nhận và hiển thị tệp tải lên trong danh sách tài liệu.
\item Người dùng nhận thông báo xác nhận và có thể xem tệp đã tải lên.
\end{itemize}

\subsection{Class Diagram}
\subsubsection{In tài liệu}
\begin{figure}[H]
    \centering
    \includegraphics[scale=0.68]{Images/class-diagram/printing.png}
    \vspace{0.1cm}
    \caption{Class diagram của nhánh chức năng in tài liệu}
\end{figure}
\begin{itemize}
    \item Sau khi chọn in tài liệu thì ở phần PrintPage, sinh viên có thể thao tác Upload(tải tài liệu), ChoosePrinter(chọn máy in) và ChangeProp(chỉnh sửa các thuộc tính in)
    \item PrintController sẽ là nơi xử lí các yêu cầu logic bao gồm RequestPrinterInfo(yêu cầu thông tin máy in), CheckFileType(kiểm tra định dạng file) và CheckProp(kiểm tra các thuộc tính in)
    \item Ở phần PrintModel sẽ lưu trữ ID, danh sách các file tài liệu, danh sách các máy in (được lấy từ method GetPrinterInfo của Database). Đồng thời có thể cập nhật các danh sách tài liệu, máy in, ... thông qua PrintModel.
    \item Từ PrintModel chúng ta sẽ sinh ra 2 class con là printerModel và fileModel:
    \begin{itemize}
        \item printerModel sẽ lưu trữ thông tin của các máy in bao gồm ID, nhãn hiệu, địa điểm, thông tin và trạng thái của máy in. Ở đây chúng ta có thể thêm, lấy hoặc set trạng thái cho các máy in.
        \item fileModel sẽ lưu trữ thông tin của các file tài liệu được tải lên bao gồm ID, định dạng tài liệu, tên, số trang, đã được kiểm tra định dạng hay chưa. Đồng thời chúng ta cũng có thể thêm, lấy hoặc set trạng thái cho các tài liệu này.
    \end{itemize}
    \item Trong trường hợp sinh viên hết số lượng giấy trong tài khoản, hệ thống sẽ chuyển hướng sang BKPay và tại đây sinh viên có thể thực hiện mua thêm giấy và sử dụng method MakePayment để thanh toán trên BKPay.
    \item Sau khi thực hiện hết tất cả các bước trên, PrintController sẽ gửi yêu cầu đến Database để thay đổi trạng thái in PrintingStatus và đưa ra thông báo trên màn hình.
\end{itemize}

\subsubsection{Quản lý máy in}
\begin{figure}[H]
    \centering
    \includegraphics[width = 1\textwidth]{Images/class-diagram/manage_printer.png}
    \newline
    \newline
    \caption{Class diagram của nhánh chức năng bật/tắt/thêm máy in}
    \label{fig:enter-label}
\end{figure}
\begin{itemize}
    \item \textit{SPSO} thực hiện các chức năng thêm/bật/tắt máy in thông qua giao diện "Quản lý máy in".
    \item \textit{PrinterManagementSystem} là nơi lưu trữ danh sách các máy in có trong hệ thống.
    \item Với nhánh chức năng bật/tắt máy in, SPSO sẽ chọn trực tiếp máy in đang hiển thị trên giao diện hoặc tìm kiếm thông qua ID, sau đó gửi yêu cầu tới hệ thống. Hệ thống sẽ tiếp nhận các yêu cầu và thực hiện các chức năng \textit{enablePrinter()}, \textit{disablePrinter()} trên máy in đã được chọn
    \item Với nhánh chức năng thêm máy in, sau khi nhận được yêu cầu từ \textit{SPSO} \textit{PrinterManagementSystem} sẽ tìm ID của máy in mới được yêu cầu thêm vào, nếu đã tồn tại, hệ thống sẽ gửi thông báo cho SPSO biết là đã có máy in này trong hệ thống, nếu không tìm thấy ID hệ thống sẽ thực hiện chức năng \textit{addPrinter()}
\end{itemize}
\newpage
\subsubsection{Tải tài liệu}
\begin{figure}[H]
    \centering
    \includegraphics[width = 1\textwidth]{Images/class-diagram/Upload.png}
    \newline
    \newline
    \caption{Class diagram của nhánh chức năng tải lên}
    \label{fig:enter-label}
\end{figure}

\begin{itemize}
    \item Sau khi chọn in tài liệu thì ở phần PrintPage, sinh viên có thể thao tác Upload(tải tài liệu)
    \item PrintController sẽ là nơi xử lí yêu cầu logic CheckFileType(kiểm tra định dạng file)
    \item Ở phần PrintModel sẽ lưu trữ ID, danh sách các file tài liệu. Đồng thời có thể cập nhật các danh sách tài liệu thông qua PrintModel.

    \item Sau khi thực hiện hết tất cả các bước trên, PrintController sẽ gửi yêu cầu đến Database để thay đổi trạng thái tải lên UploadingStatus và đưa ra thông báo trên màn hình.
\end{itemize}

\subsection{Develop MVP 1 as user interfaces of either a Desktop-view central 
dashboard for a particular module (the same with the module used in task 
2.1). Decide yourself what to include in the view. Use a wireframe tool like 
Figma or Adobe XD, or Illustrator}
\subsubsection{Default page}
\begin{figure}[H]
    \centering
    \includegraphics[width = 1\textwidth]{2.4/default_page.png}
    \newline
    \newline
    \caption{Trang chủ trước khi đăng nhập}
    \label{fig:enter-label}
\end{figure}

\subsubsection{Homepage}
\begin{figure}[H]
    \centering
    \includegraphics[width = 1\textwidth]{2.4/Trang chu.png}
    \newline
    \newline
    \caption{Trang chủ}
    \label{fig:enter-label}
\end{figure}

\subsubsection{Sign up}
\begin{figure}[H]
    \centering
    \includegraphics[width = 1\textwidth]{2.4/Dang ki.png}
    \newline
    \newline
    \caption{Trang đăng ký}
    \label{fig:enter-label}
\end{figure}

\subsubsection{Log in}
\begin{figure}[H]
    \centering
    \includegraphics[width = 1\textwidth]{2.4/Dang nhap.png}
    \newline
    \newline
    \caption{Trang đăng nhập}
    \label{fig:enter-label}
\end{figure}

\subsubsection{Printing}
\begin{figure}[H]
    \centering
    \includegraphics[width = 1\textwidth]{2.4/In tai lieu.png}
    \newline
    \newline
    \caption{Trang in tài liệu}
    \label{fig:enter-label}
\end{figure}

\subsubsection{Printing history}
\begin{figure}[H]
    \centering
    \includegraphics[width = 1\textwidth]{2.4/Lich su in.png}
    \newline
    \newline
    \caption{Trang xem lịch sử in}
    \label{fig:enter-label}
\end{figure}

\subsubsection{Printer management}
\begin{figure}[H]
    \centering
    \includegraphics[width = 1\textwidth]{2.4/Quan li may in.png}
    \newline
    \newline
    \caption{Trang quản lý máy in}
    \label{fig:enter-label}
\end{figure}

